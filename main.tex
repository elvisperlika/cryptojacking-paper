\documentclass[12pt,a4paper]{article}
\usepackage{textcomp, gensymb}
\usepackage[italian]{babel}
\usepackage{newlfont}
\usepackage{gensymb}
\usepackage{hyperref}
\usepackage{graphicx}
\usepackage{mathtools}
\usepackage{amssymb}

\textwidth=450pt\oddsidemargin=0pt
\begin{document}
\begin{titlepage}
\begin{center}
\rule[0.5cm]{15.8cm}{0.6mm}
{\small{\bf Relazione di Crittografia }}
\end{center}
\vspace{15mm}
\begin{center}
{\LARGE{\bf Cryptojacking:} \\ 
\vspace{3mm}
{\bf quando il tuo Computer lavora per qaualcun altro.}}
\end{center}
\vspace{35mm}
\par
\noindent
\begin{center}
{\large{\bf Elvis Perlika}}
\end{center}
\begin{center}
{\large{\bf 0000970373}}
\end{center}
\hfill

\vspace{70mm}
\begin{center}
{\large{\bf Corso di Crittografia \\ 
A.A. 2023-2024 \\
Prof. Luciano Margara}}
\end{center}
\end{titlepage}

\newpage

\tableofcontents

\newpage

\section{Introduzione}
\subsection{Definizione}

Il Cryptojacking, in Italiano "Dirottamento di risorse", è una forma di attacco
informatico che sfrutta la potenza di calcolo di un utente, senza che esso ne
sia consapevole, per minare criptovalute. \cite{CSO}

Gli Hacker hanno come obbiettivo quello di prendere il controllo del maggior
numero possibile di sistemi con l'obbiettivo di minare quante più criptovalute,
illecitamente. Questo sistema di hacking non punta unicamente la classica utenza
di Personal Computer ma cerca di sfruttare anche le risorse di Server e
infrastrutture Cloud e in generale ogni tipologia di sistema computazione con un
accesso alla rete Internet.

La caratteristica fondamentale di questo malware è far sì che la vittima sia
ignara dei processi in background, che si occupano di minare, e permettergli di
usare la propria macchina normalmente. Ovviamente, il tutto, a discapito di un
sovraccarico della macchina e conseguente surriscaldamento, presenza di lag,
maggior consumo elettrico (che nel caso di servizi Server o Cloud porta ad avere
fatture particolarmente elevate) e riduzione delle performance generali.

Questo paper si propone di analizzare il fenomeno del cryptojacking, i metodi di
attacco, le tecnologie coinvolte e i relativi aspetti tecnici per poi esporre le
contromisure per prevenire e individuare questi malware al intero delle proprie
macchine. In particolare, nella sezione \hyperref[sec:aspetti_tecnici]{"Aspetti
Tecnici"} si andarà ad analizzare nel dettaglio il mining di Monero, una delle
criptovalute più utilizzate per il cryptojacking e l'algoritmo CryptoNight,
utilizzato per minare Monero.

\subsection{Storia}

Una delle prime forme di cryptojacking è stata scoperta nel Giugno 2011, quando
l'azienda Symantec Corporation iniziò a sospettare che le botnet \footnote{ Una
botnet è un gruppo di dispositivi connessi a Internet , ognuno dei quali esegue
uno o più bot . Le botnet possono essere utilizzate per eseguire attacchi DDoS (
distributed denial-of-service ), rubare dati, [1] inviare spam e consentire
all'aggressore di accedere al dispositivo e alla sua connessione. Il
proprietario può controllare la botnet utilizzando un software di comando e
controllo (C\&C). [2] La parola "botnet" è una parola risultata dalla unione
delle parole " robot " e " network ". Il termine è solitamente utilizzato con
una connotazione negativa o malevola.\cite{Botnet}} potessero minare Bitcoin
segretamente, sebbene la GPU di una sola macchina impiegherebbe molto tempo per
minare una transizione in criptovalute, utilizzando una grande quantità di
macchine si riesce a suddividere il lavoro e ridurre il tempo.

Una serie di attacchi rilevanti di cryptojacking sono stati scoperti dal 2011 al
2021. L'ultimo è relativo al "2021 Microsoft Exchange Server data breach"
\cite{zero-day}, tale breccia, creata nel Gennaio 2021 ha permesso numerosi
attacchi tra qui diversi di tipo cryptojacking.

Il cryptojacking è emerso come una minaccia significativa nel campo della
cybersecurity intorno al 2017, con l'introduzione di Coinhive, dismesso poi a
Marzo 2019 era un servizio di mining di criptovalute attraverso i browser web,
che andava a utilizzare parte o tutta la potenza di calcolo per minare
criptovalute Monero (approfondimento nella sezione
\hyperref[sec:aspetti_tecnici]{"Aspetti Tecnici"}). 

\subsection{Struttura del documento}
Fornisci una panoramica della struttura del documento, indicando brevemente cosa
verrà trattato nelle diverse sezioni.

\subsection{Importanza}
Spiega perché l'argomento è rilevante o significativo. Questo aiuta a
contestualizzare la tua discussione e a motivare l'interesse del lettore.

\subsection{Problema e Motivazione}
Identifica il problema specifico che il tuo documento cerca di risolvere e
spiega perché è importante affrontarlo.

\newpage

\section{Come funziona}
Il mining, cioè il processo che Bitcoin e altre cripto valute utilizzano per
coniare virtualmente nuove monete digitali e certificare le transazioni, usando
le relative monete, è completamente lecito. 

Nel dettaglio troviamo vaste reti decentralizzate di computer in tutto il mondo
che verificano e proteggono le blockchain, ovvero i registri virtuali che
documentano le transazioni di criptovalute. In cambio del contributo della loro
potenza di elaborazione, l'utente del computer della rete che per primo risolve
i calcoli complessi dovuti alla certificazione della transizione viene premiato
con nuove monete. Si tratta di un circolo virtuoso: i minatori mantengono e
tutelano la blockchain, la blockchain assegna le monete, le monete fungono da
incentivo ai minatori per continuare a mantenere la blockchain. Il mining è
l'unico modo per rilasciare nuove cripto monete nella rete ed è un processo che
richiede molta potenza di calcolo con un effort inversamente proporzionale al
mining effettuato portando così ad un aumento della difficoltà di mining e ad
una conseguente crescita dei costi.

Il cryptojacking sfrutta questo processo, ma in modo illecito. Gli hacker
inseriscono codice malevolo nei siti web o nei messaggi di posta elettronica che
infettano i computer delle vittime e li trasformano in macchine per il mining
riducendo i costi e aumentando i guadagni.

\newpage

\section{Metodi di attacco}
I metodi per attaccare un sistema con il cryptojacking sono molteplici e variano
a seconda del tipo di sistema che si vuole attaccare. I metodi più comuni sono:

\subsection{Attaccare diretamente i Personal Computer}
Attaccare uno o più PC è il classico metodo per creare un sistema di
cryptojacking. Tipicamente l'hacker riesce ad iniettare il suo software di
mining all'interno della macchina usando tecniche come:
\begin{itemize}
    \item \textbf{Fileless malware}: possono essere di 2 tipologie:
    \begin{itemize}
        \item \textit{Fully Fileless Malware}: non esegue nessun file sul disco
        ma tutte le attività possono essere osservate in memoria. Gli hacker
        possono anche, attraverso la rete, inivare pacchetti malevoli che
        installano backdoor che risiedono nella memoria kernel.
        \item \textit{Fileless Malware with Indirect File Activity}: non scrive
        direttamente i file sul disco, ma gli autori delle minacce possono
        installare un comando PowerShell all'interno del repository WMI
        configurando un filtro WMI per la persistenza. Anche se in teoria
        l'oggetto WMI dannoso esiste su un disco, non tocca il file system sul
        disco. Si tratta quindi di un attacco senza file poiché, secondo
        Microsoft [34], "l'oggetto WMI è un contenitore di dati multiuso che non
        può essere rilevato e rimosso".\cite{FMW}
    \end{itemize}
    \item \textbf{Schemi di phishing}: è il modo più semplice con cui gli
        aggressori di cryptojacking possono rubare risorse è inviare agli utenti
        un'e-mail dall'aspetto legittimo che li incoraggi a fare clic su un
        collegamento che esegue il codice per inserire uno script di
        cryptomining sul proprio computer. Funziona in background e invia i
        risultati tramite un'infrastruttura di comando e controllo
        (C2\footnote{Command and Control Infrastructure: anche conosciuto come
        C\&C o C2 è il set di strumenti e tecniche che un un hacker utilizza per
        mantenere la commuicazione con il computer precedentemente compresso.}).
    \item \textbf{Embedded di script malevoli al interno di siti o web app}: gli
        hacker possono sfruttare script all'interno dei siti, che eseguiti
        automaticamente dai browser, minano le cripto valute. Questo metodo è
        molto più diffuso e meno invasivo rispetto al precedente, poiché non
        sscarica alcun codice nel dispositivo.
\end{itemize}

\subsection{Cercare server e dispositivi di rete vulnerabili}
I server sono un obbiettivo molto ambito per gli hacker, in quanto sono
dispositivi molto potenti e spesso connessi a Internet 24/7. Gli hacker possono
sfruttare vulnerabilità come Log4J\footnote{"La vulnerabilità Log4j, conosciuta
anche come Log4Shell, è una vulnerabilità critica scoperta nella libreria di
registrazione Apache Log4j nel novembre del 2021. Sostanzialmente, Log4Shell
concede agli hacker il controllo totale dei dispositivi eseguendo versioni di
Log4j senza patch." - \href{https://arc.net/l/quote/zjujxamu}{IBM}} per
iniettare i propri sistemi di cryptojacking in queste potenti macchine. Spesso i
server compromessi vengono anche utilizzi come potente per accedere con maggior
semplicità ad altri dispositivi per eseguire attacchi più complessi ed
orizzontali.

\subsection{Attaccare il sistema di produzione di software}
Un altro metodo molto comune è quello di attaccare il sistema di seminare
repository open-souce nelle quali è stato iniettato il loro codice malevolo.
Grazie ai programmatori che utilizzano questi codici è possibile per gli hacker
raggiungere un numero elevato di macchine e scalare velocemente il loro sistema
di mining. Una volta entrati nella macchina del programmatore, possono cercare
di accedere anche ai server, ai dispositivi di rete oppure ai servizi cloud ai
quali esso è connesso. In alternativa possono puntare a sub-iniettare questi
script all'interno dei progetti che i programmatori stanno sviluppando.

\subsection{Fare leva sulle infrastrutture cloud}
Come per i server, anche le infrastrutture cloud sono un obbiettivo molto ambito
poiché permettono di effettuare computazioni ancora più veloci. Uno dei metodi
più comuni per farlo è scansionare le API dei container esposti e utilizzare
tale accesso per avviare il caricamento del software di mining sulle istanze dei
container o sui server cloud interessati. L'attacco è in genere automatizzato
con un software di scansione che cerca server accessibili alla rete Internet
pubblica con API esposte o che permettono l'accesso senza autenticazione. Come
per i server, gli aggressori sfruttano il cloud service violato ed attraverso lo
stesso puntano a raggiungere altre infrastrutture simili. Questi sono gli
attacchi più redditizi. \\
L'aspetto rilevante, in tutti gli approcci sopra citati, è che gli hacker
possano accedere a quante più macchine computazionali.

\newpage 

\section{Aspetti tecnici di Monero}\label{sec:aspetti_tecnici} Non è obbiettivo
di questo paper approfondire il tema delle criptovalute in senso generale ma si
vuole trattare il tema del mining in modo più specifico. Nella seguente sezione
si andrà ad analizzare il mining di Monero, una delle criptovalute più
utilizzate per il cryptojacking.

La criptovaluta Monero, inizialmente nota come BitMonero, è stata creata
nell'aprile 2014 come deriva della valuta proof-of-concept CryptoNote. Monero
significa "denaro" nella lingua esperanto.

CryptoNote è una criptovaluta ideata da vari individui. Un white paper di
riferimento che lo descrive è stato pubblicato sotto lo pseudonimo di Nicolas
van Saberhagen nell'ottobre 2013. Grazie a CryptoNote e al suo algoritmo di
hashing, CryptoNight, Monero è diventata una delle criptovalute più popolari per
il mining.

Una delle filosofie di Monero è quella di mantenere un mining egualitario, in
modo che tutti possano avere la possibilità di fare mining. Per raggiungere
questo obiettivo, utilizza un algoritmo particolare ideato e sviluppato dai
membri della community della criptovaluta: RandomX . Questo algoritmo PoW è
resistente agli ASIC, il che rende impossibile costruire hardware specializzato
per fare mining di Monero. I miner sono obbligati ad utilizzare hardware di
livello consumer e competere lealmente.

\subsection{Fondamentali}
Le curve ellittiche sono la funzione matematica che sta alla base della
crittografia delle criptovalute. Queste curve sono utilizzate per creare le
chiavi pubbliche e private che permettono di firmare e verificare le
transizioni. Procediamo con criterio per capire come funzionano le curve
ellittiche, questo sarà fondamentale per comprendere il funzionamento di Monero
e delle sue caratteristiche di privacy.

\subsubsection{Aritmetica Modulare}
L'aritmetica modulare, detta anche \textit{Aritmetica dell'orologio}, è un
sistema di aritmetica degli interi, in cui i numeri "si avvolgono su loro
stessi" ogni volta che raggiungono i multipli di un determinato numero $ n $,
detto \textbf{modulo}.

Inconsciamente utilizziamo l'aritmetica modulare ogni volta che guardiamo un
orologio. Ad esempio, se sono le 10:00 e aggiungo 3 ore, il risultato sarà 1:00
e non 13:00. Questo perché l'orologio è un sistema di 12 ore, quindi il modulo è
12; questo è il motivo per cui viene chiamata \textit{aritmetica dell'orologio}.

Diciamo che per calcoalre $ c = a \mod{b} $ possiamo immaginare un asse di
numeri interi e posizionarci si $ a $ e 'saltare' con passi di lunghezza $ b $
fino a raggiungere un valore intero che sia $ \ge 0 $ e $ < b $, questo sarà il
nostro $ c $. Ad esempio:
$$ -5 \mod{3} = 1 \qquad \text{oppure} \qquad 4 \mod 3 = 1 $$

Formalmente possiamo definire l'equazione $ c = a \mod{b} $ come $ a = bx + c $
dove $ x $ è il quoziente e $ c $ è il resto di $ a \mod b $.

Ne seguono alcune proprietà che verranno definite in seguito.

\subsubsection{Curve Ellittiche}
Definiamo una curva ellittica $ E $ su un campo finito $ F_p $ dove $ p $ è un
numero primo a 256 bit e la presentiamo in forma di Weierstrass come:

$$ E: y^2 = x^3 + ax + b \enspace | \enspace x, y \in F_p $$

in cui $ a $ e $ b $ sono i parametri della curva che ne defiscono la forma e la
posizione. Le coordinate $ (x,y) $ sulla curva ellittica che possono prendere
qualsiasi valore all'interno di $F_p$ formano un Gruppo Abeliano \footnote{Un
gruppo abeliano è un gruppo in cui l'operazione beneficia della proprietà
commutativa È anche detto: Gruppo Commutativo}. Questo particolare gruppo ci
permette, scegliamo 2 punti $ P $ e $ Q $ sulla curva che useremo per risolvere
$ R $ andando a eseguire l'operazione di somma $ P + Q = R $ con $ R $ che sarà
un altro punto sulla curva.

Prendiamo gli scalari $p, q$ valori interi random di grandezza $n$ tali che $ p,
q \in {o,1}^n $.

Il Standards for Efficient Cryptography (SEC) è un set di curve ellittiche
proposte per l'uso nel campo della crittografia. Una delle più note e utilizzate
è la \textbf{Secp256k1} definita dalla equazione
$$ y^2 = x^3 + 7 \mod{p} $$ dove 
$$ p = 2^{256} - 2^{32} \underbrace{- 2^9 - 2^8 - 2^7 - 2^6 - 2^4 - 1}_{-977}
$$.

Questa curva è la base per la crittografia di Bitcoin e altre criptovalute.
Questa funzione possiede diverse qualità tali che è stata applicata, non solo
nel abito delle criptovalute, ma anche in altri campi per rendere le
commuicazione sicure; come quello del IoT. Monero, invece, utilizza la curva
Ed25519, di cui parleremo più avanti.

\subsection{Scambio di chiavi Diffie-Hellman con curve ellittiche}
Il protocollo di scambio di chiavi Diffie-Hellman (DH), inventato nel 1976, nato
dalla collaborazione dei ricercatori Whitfield Diffie e Martin Hellman, è il
primo protocollo a permettere a 2 parti di communicare attraverso un canale
insicuro senza necessità di condiversi una chiave segreta previa commuicazione.
Vennero così introdotti i cifrari a chaive pubblica. Matematicamente, il
protocollo DH si basa sul problema della fattorizzazione e sul problema del
logaritmo discreto nell'algebra modulare.

Definiamo $ p $ un numero reale random tale che $0 < k < l$ che chiameremo
\textit{private key} e calcoliamo la relativa \textit{public key} $ P = k \cdot
G $ dove $ G $ è il generatore del gruppo abeliano in questione.

Un classico scambio di segreti tra Bob e Alice, utilizzando le curve ellittiche,
avviene nel seguente modo:

\begin{enumerate}
    \item Alice e Bob generano le proprie chiavi pubbliche e private $ (p_A,
    S_A) $ e $ (p_B, S_B) $ rispettivamente. Entrambi condividono le proprie
    chiavi pubbliche ma non quelle private.
    \item Assumendo
        $$ X = p_A \cdot S_B = p_A \cdot p_B \cdot G = p_B \cdot p_A \cdot G =
        p_A \cdot S_A $$

        Alice e Bob dovranno calcolarsi, privatamente: $ X = p_A \cdot S_B $ e $
        X = p_B \cdot S_A $. Queste saranno le chiavi condivise.

        Un osservatore esterno non riuscirà a calcolare $ S $, cioè il segreto,
        in modo semplice proprio a causa del problema di Diffie-Hellman. Infatti
        trovare $ S $ a partire da $ S_A $ e $ S_B $ è un problema
        computazionalmente estremamente difficile.
\end{enumerate}

\subsection{Shnorr Signature}
In crittografia, per Shnorr Signature si intende l'algoritmo di firma digitale
ideato da Clauss Schnorr nel 1989. È uno dei primi protocolli basati sulla
\textbf{impraticibiltà} nel risolvere il problema del logaritmo discreto, la sua
funzione è permette di dimostrare ad una delle 2 parti in comunicazione di
conoscere la chiave privata relativa a quella pubblica senza rivelare, appunto,
quella privata. Questo algoritmo ci sarà utile per quando tratteremo le Ring
Signature (tradotte: Firme ad Anello).

\subsubsection*{Algoritmo}

È fondamentale che tutti gli utenti della commuicazione concordino sul gruppo
abeliano $ G $, di ordine $ q $, generato da $ g $ e per assunzione in questo
gruppo il problema del logaritmo discreto sia molto difficile. Oltre al gruppo
devono conconrdare anche su una funzione di hash sicura $ H:\{0,1\}^*
\rightarrow \mathbb{Z} / q\mathbb{Z} $.

\paragraph{Generazione delle chiavi}
Si sceglie una chiave privata $ p \in \mathbb{Z}_q $ e si calcola la chiave
pubblica $ S = g^{-p} $.

\paragraph{Firma}
Per firmare un messaggio $ m $ si procede nel seguente modo:
\begin{enumerate}
    \item si genera un numero reale random $ k \in \mathbb{Z}_q $
    \item si calcola $ x = g^k $ con $ x \in G $ 
    \item si calcola $ r = H(x || m) $ dove $ || $ rappresenta la concatenazione
    in stinghe di bit
    \item si calcola $ c = k + p \cdot r $ con $ e \in \mathbb{Z}_q $
\end{enumerate}

Abbiamo, così, creato la firma $ (c, r) $ per il messaggio $ m $. Chiameremo $ c
$ la 'challenge' e $ r $ la 'response'.

\paragraph{Verifica}
Per verificare la firma $ (c, r) $ si procede nel seguente modo:
\begin{enumerate}
    \item si calcola $ x_v = g^c\cdot y^r $
    \item si calcola $ r_v = H(x_v || m) $
\end{enumerate}
Se $ r_v = r $ allora la firma è valida.

\paragraph{Dimostrazione}
$$ x_v = g^c\cdot y^r = g^{k + p\cdot r}\cdot g^{-p\cdot r} = g^k = x $$ ed in
seguito:
$$ r_v = H(x_v || m) = H(x || m) = r $$ quindi il messaggio firmato corrisponde
a quello verificato.

Anche se un intruso, senza conoscere la chiave privata, avesse creato la firma
(c, r) sarebbe stato trascurabile, quindi un verificatore può essere sicuro che
il messaggio non sia stato manomesso.

\subsection{Curva Ellittica Ed25519}
Ed25519 è una particolare \textit{Twisted Edwards elliptic curve} che utilizza
Monero per le operazioni crittografiche. La curva è definita dal campo $
F_{2^{255} - 19} $ e dalla curva ellittica:
$$ -x^2 + y^2 = 1 - \frac{121665}{121666}x^2y^2 $$

La communità scientifica (il NIST) pensa che questa curva non sia così sicura e
affidabile. 

Le curve Twisted Edwards sono di ordine $ N = 2^c l $ con $ l $ numero primo e $
c $ un intero positivo. Nel caso di Ed25519 il suo ordine è di 76 cifre e quindi
$ l $ è a 253 bits.
$$ l = 2^3 \cdot
7237005577332262213973186563042994240857116359379907606001950938285454250989 $$

Il campo $ F_{2^{255} - 19} $ è codificato in 32 byte, ovvero 256 bit. Di
conseguenza, qualsiasi punto in Ed25519 potrebbe essere espresso utilizzando 64
byte poichè includono sia una rappresentazione del punto $ R $ che un valore
scalare $ S $ derivato da una computazione su una funzione di hash $ H $:
$$ S = (H + \textit{private key} \times H) \mod l $$ Applicando le tecniche di
point compression, descritte di seguito, tuttavia, è possibile ridurre questa
quantità della metà, a 32 byte utilizzando tecniche di "Point Compression"
\footnote{ Le tecniche di point compression (compressione dei punti) sono metodi
utilizzati in crittografia ellittica per ridurre la quantità di dati necessari
per rappresentare un punto su una curva ellittica. Questo è particolarmente
utile per ridurre l'uso di memoria e banda, specialmente in applicazioni che
richiedono l'invio o la memorizzazione di grandi quantità di punti su curve
ellittiche, come nelle firme digitali o nei protocolli di scambio di chiavi.}.

\subsection{Shnorr Signature Avanzato}
In Shnorr base utilizziamo una sola chiave ma possiamo rendere Shnorr più
sofisticato utilizzando più chiavi. Questo è il caso dello schema Multi Layer
Linkable Spontaneous Anonymous (MLSAG), uno schema di firma che permette a più
utenti di firmare un messaggio in modo anonimo. 

È spesso vantaggioso dimostrare che la stessa chiave privata è stata utilizzata
per generare chiavi pubbliche su basi diverse. Per esempio, consideriamo una
chiave pubblica standard $kG$ e un segreto condiviso di Diffie-Hellman $kR$ con
la chiave pubblica di un'altra persona, dove le basi sono rispettivamente $G$ e
$R$. In questo contesto, possiamo dimostrare la conoscenza del logaritmo
discreto $k$ relativo a $kG$, provare la conoscenza di $k$ in $kR$, e confermare
che $k$ è identico in entrambe le situazioni, senza tuttavia rivelare il valore
di $k$.

È possibile trovare una \textit{Non-interactive proof\footnote{}} a pagina 25
del white paper di Monero \cite{Zero To Monero}.

Chiamiamo Non-Interactive Proof è un metodo crittografico in cui una parte (il
provatore) può dimostrare a un'altra parte (il verificatore) che una certa
affermazione è vera senza interagire direttamente con il verificatore durante il
processo di prova, diversamente dal tradizionale \textit{Zero-knowledge proofs}
che necessità di una interazione tra le parti simultanea.

\subsection{Privacy in Monero}

Monero può essere estratto sia da CPU che da GPU, ma la prima è molto più
efficiente. E' evidente che sia la cripto valuta più pratica per il
cryptojacking, poiché può essere minata solo su macchine a livello consumer, le
quali sono faccilemente accessibili da cyber-criminali attraverso i metodi
precedentemente citati. Inoltre, utilizza una blockchain\footnote{Libro
contabile digitale condiviso in rete, è il sistema fondamentale di una
criptovaluta in quanto tiene memoria di tutte le transizioni eseguite nella
storia della realtiva criptovaluta. Viene detta blockchain poiché è una catena
di blocchi, ognuno rappresenta una transizione che viene agganciata alla catena
attraverso la risoluzione di calcoli complessi (mining).} supportata da un
\textbf{Privacy-enhancing technologies} sofisticato, il quale derivava da
CryptoNight. Al fine di fornire privacy e anonimato, Monero, si basa su due
concetti importanti: Stealth Address e Ring Signature.

\subsubsection{Address}
In tutte le blockchain, per ogni utente viene generato un indirizzo, questo è
tutto ciò che serve per ricevere pagamenti. Poichè il libro mastro è pubblico,
tutti possono vedere gli indirizzi e le transizioni e si può facilmente
comprendere per ogni indirizzo l'ammontare di cripto valuta che possiede.

Monero, diveresamente da altre cripto valute come BitCoin, utilizza nella
transazione due coppie di chiavi private/pubbliche: $(k^v, K^v)$ e $(k^s, K^s)$.
La seconda coppia è l'indirizzo del utente, mentre la prima è la corrispondente
chiave privata. Indichiamo con $k^v$ le \textit{view key} e con $k^s$ la
\textit{spend key}.

La chiave di visualizzazione viene utilizzata per verificare se un'uscita è
associata al proprio indirizzo, mentre la chiave di spesa consente di "spendere"
quell'uscita in una transazione per poi confermare che è stata spesa.

Di seguito spiegherò in sintesi alcune scelte di design di Monero per garantire
la massima riservatezza e anonimato nelle transizioni. Se si vuole approfondire
si può consultare il white paper di Monero \cite{Zero To Monero} nelle pagine
37-42.

\paragraph{One-time address}
Se, generalmente, un utente deve condividere il proprio indirizzo per ricevere
pagamenti, Monero utilizza un sistema di indirizzi monouso. Come facciamo a
condividere un indirizzo monouso? Utilizzando uno scambio di tipo
Diffie-Hellman, così anche un osservatore che conosce tutti gli address non può
comprendere chi stia eseguendo la transizione e verso chi.

Facciamo un esempio: Alice vuole inviare 10 Monero a Bob. Bob ha le proprie
coppie di chiavi $(k^v_B, K^v_B)$ e $(k^s_B, K^s_B)$ e Alice conosce le chiavi
pubbliche di Bob quindi il suo inidirizzo.

Parafrasando Butrin\footnote{Vitalik Buterin, co-fondatore di Etherium}:
\begin{quote}
    Sia il destinatario (chiamiamolo "Bob") che il mittente ("Alice") possono
    generare un indirizzo invisibile per la transazione. Tuttavia, solo il
    destinatario, Bob, può controllare la transazione. Un altro modo di pensare
    a un indirizzo invisibile è come un indirizzo di portafoglio legato
    crittograficamente all’indirizzo pubblico di Bob, ma che viene rivelato solo
    alle parti che effettuano la transazione. \cite{Buterin Quote}
\end{quote}

Questi One-time address sono anche detti Stealth Address. Il team di Buterin ha
progettato un sistema di indirizzi nascosti (anche detto SAP \footnote{Stealth
Address Protocol}) chiamato BaseSAP. Il protocollo mira a fornire un meccanismo
leggero che consenta agli utenti di generare indirizzi temporanei, mantenendo la
completa compatibilità con le versioni precedenti e non richiedendo modifiche
alla blockchain principale. BaseSAP è basato sul cifrario assimetrico su curve
ellittiche Secp256k1, migliorato attraverso l'integrazione di "tags di
visualizzazione" utili a rendere più efficiente l'analisi rispetto ai comuni
DKSAPs \footnote{Dual-Key Stealth Address Protocols}.

Questo protocolli sono la base per le implementazioni DKSAP usate in Monero. Da
quando DKSAP è nato, ha portato molti ricercatori a studiare e trovare nuovi
modi per migliorarlo:

\begin{figure}[ht]
    \centering
    \includegraphics[width=0.95\textwidth]{./images/sommario.png}
    \caption{Sommario dei lavori di ricerca sulle Stealth Address e compatibilità BaseSAP}
    \label{fig:summary}
\end{figure}l

\begin{enumerate}
    \item Alice genera una reale random $r \in \mathbf{Z}l $ e calcola il
    One-time address $$ K^{o} = \mathcal{H}_n(rK^v_B)G+K^s_B $$ e definisce
    $K^{o}$ come l'indirizzo di Bob, al quale specifica l'importo di 10 Monero
    ed il valore $rG$ e pubblica il tutto sulla blockchain.
    \item Bob, una volta ricevuti i dati, calcola $k^v_BrG = rK^v_B$ e calcola
    ${K'}^s = K^o - \mathcal{H}_n(rK^v_B)G$. Una volta che ha ha compreso se
    $K'^s = K^s$ comprendo che l'outout è per lui.
    \item una volta che Bob avrà confermato che l'output è per lui utilizzando
    la sua \textit{view key}, potrà firmare un messaggio con la sua
    \textit{spend key} e inviarlo alla rete per dimostrare di essere Bob e
    ricevere i Monero accordati.
\end{enumerate}

\paragraph{Subaddress}
Un'altra tecnica per garantire la privacy è quella di utilizzare gli
\textit{subaddress}. Ogni utente può generare dei subaddress partendo dal
proprio address. Questi subaddress possono essere utilizzati per ricevere
pagamenti, possiamo immaginare l'address come una 'Banca' e i subaddress come i
relativi 'Bancomat'. Oltre ad essere utili per aumentare la privacy, l'utente
che crea i propri subaddress può utilizzarli per distinziare una transazione da
un'altra.

\subsubsection{Amount Hiding}

\subsubsection{Ring Signature}

\newpage 

\section{Popolarità}
La popolarità è dovuta al potenziale guadagno, guadagno molto facile da crearsi
poiché per definizoione il cryptojacking punta a sfruttare risorse in possesso
di altri in modo gratuito. Così, anche considerando la volatilità delle cripto
valute, esempio principe BitCoin, i margini di guadagno sono abbastanza alti da
rendere il crimine un vero e proprio business. 

\newpage

\section{Prevenire e individuare}

\newpage

\section{Casi reali}

\subsection{WatchDog targets Docker Engine API endpoints and Redis servers}
Un gruppo di hacker, chiamato WatchDog, ha attaccato i server
Docker\footnote{Docker è una piattaforma software che permette di creare,
testare e distribuire applicazioni con la massima rapidità. Docker raccoglie il
software in unità standardizzate chiamate container che offrono tutto il
necessario per la loro corretta esecuzione, incluse librerie, strumenti di
sistema, codice e runtime. - AWS} e Redis\footnote{Redis, un sistema di gestione
di database NoSQL lanciato nel 2009, utilizza un modello di archiviazione basato
su coppie chiave/valore. Ogni dato è memorizzato in un dizionario, in cui una
chiave univoca è associata a un valore specifico, rendendo semplice il recupero
delle informazioni.}. Il gruppo riusciva ad infiltrarsi nei Docker Engine API
attraverso la porta 2375 aperta, una volta dentro, gli intrusi, potevano
accedere alla Shell di comando.

\paragraph{Payload} Viene caricato nel container uno script \textbf{cronb.sh}
che va a controllare lo stato del container ed eventualmente eseguire un secondo
script \textbf{ar.sh}, il quale va sabatore completamente il container e
caricare un miner XMRRig\footnote{Un software di mining di Monero}. Un ultimo
Payload è una serie di script che gli intrusi usano per puntare ad altri sisteme
collegati alla rete del container.\cite{WatchDog}

\subsection{Alibaba ECS instances in cryptomining crosshairs}
Come nel caso di Docker, anche in questo caso gli hacker hanno puntato alle
sfrutture cloud, ma di Alibaba. I gruppi in question esembrerebbero essere
TeamTNT, Kinsing ed altri.

I server Alibaba ECS\footnote{Elastic Compute Service} sono forniti con un preinstallato 
agente di sicurezza.  

Di seguito un codice specifico del malware che crea
regole firewall per eliminare i pacchetti in entrata da intervalli IP
appartenenti a zone e regioni interne di Alibaba.

\begin{figure}[ht]
    \centering
    \includegraphics[width=0.95\textwidth]{./images/alibaba_firewall.png}
    \caption{Codice dannoso che modifica le regole del firewall}
    \label{fig:alibaba_firewall}
\end{figure}

\begin{figure}[ht]
    \centering
    \includegraphics[width=0.95\textwidth]{./images/alibaba_agent_disable.png}
    \caption{Script che disabilità l'agente di sicurezza di Alibaba}
    \label{fig:alibaba_agent}
\end{figure}

Quando un malware di cryptojacking è attivo su un'istanza Alibaba ECS, l'agente
di sicurezza installato, se lo rileva, notifica la presenza di uno script
dannoso. A quel punto, è compito dell'utente, gestore della piattaforma,
intervenire per arrestare l'infezione e le attività malevole. Alibaba Cloud
Security fornisce indicazioni su come procedere, ma la responsabilità principale
dell'utente rimane quella di prevenire l'infezione fin dall'inizio.


\newpage

\section{Bibliografia}
\begin{thebibliography}{9}

\bibitem{CSO}
Ericka Chickowski, \href{https://arc.net/l/quote/karbftmg}{Cryptojacking
explained: How to prevent, detect, and recover from it, CSO}, 20 Giugno, 2022

\bibitem{Botnet}
\href{https://arc.net/l/quote/ftyxgxms}{Botnet, Wikipedia}
% \texttt{31-07-2024 13:00}

\bibitem{zero-day}
\href{https://arc.net/l/quote/golshtco}{2021 Microsoft Exchange Server data breach, Wikipedia}
% \texttt{01-08-2024 16:00}

\bibitem{Monero}
\href{https://arc.net/l/quote/jffmkeln}{Monero, Wikipedia}
% \texttt{01-08-2024 16:30}

\bibitem{FMW}
\href{https://arc.net/l/quote/kbarlqni}{Fileless malware, IEEE Xplore}

\bibitem{Buterin Quote}
Valerio Diaco, \href{https://arc.net/l/quote/qirrmtbh}{\textit{Conosci gli
Stealth Address per star lontano dai radar?}}, Rypto.it, 15 Lugio 2023

\bibitem{BaseSAP}
Anton Wahrstatter , Matthew Solomon, Ben DiFrancesco, Vitalik Buterin, and Davor
Svetinovic \textit{BaseSAP: Modular Stealth Address Protocol for Programmable
Blockchains}, JOURNAL OF LATEX CLASS FILES, VOL. 14, NO. 8, Agosto 2021, pp. 1–6

\bibitem{Zero To Monero}
Koe, Kurt M. Alonso, Sarang Noether, \textit{Zero to Monero: Second Edition - A
technical guide to a private digital currency; for beginners, amateurs, and
experts} 4 Aprile 2020 (v2.0.0)

\bibitem{Margara}
Margara Luciano, \textit{Diffie Hellman, Crittografia su Curve Ellittiche} A.A.
2023-2024

\bibitem{EdDSA}
Simon Josefsson and Ilari Liusvaara. Edwards-Curve Digital Signature Algorithm
(EdDSA). RFC 8032, Gennaio 2017. https://rfc-editor.org/rfc/rfc8032.txt

\bibitem{WatchDog}
\href{https://cyware.com/news/watchdog-targets-docker-and-redis-servers-in-new-cryptojacking-campaign-a5681a92}{WatchDog
Targets Docker And Redis Servers In New Cryptojacking Campaign}, 06 Giugno 20222

\bibitem{Alibaba}
David Fiser, Alfredo Oliveira,
\href{https://www.trendmicro.com/en_us/research/21/k/groups-target-alibaba-ecs-instances-for-cryptojacking.html}{Groups
Target Alibaba ECS Instances for Cryptojacking}, 15 Novembre 2021

\end{thebibliography}


\end{document}